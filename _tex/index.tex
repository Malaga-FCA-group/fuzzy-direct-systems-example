% Options for packages loaded elsewhere
\PassOptionsToPackage{unicode}{hyperref}
\PassOptionsToPackage{hyphens}{url}
\PassOptionsToPackage{dvipsnames,svgnames,x11names}{xcolor}
%
\documentclass[
  letterpaper,
  DIV=11,
  numbers=noendperiod]{scrartcl}

\usepackage{amsmath,amssymb}
\usepackage{iftex}
\ifPDFTeX
  \usepackage[T1]{fontenc}
  \usepackage[utf8]{inputenc}
  \usepackage{textcomp} % provide euro and other symbols
\else % if luatex or xetex
  \usepackage{unicode-math}
  \defaultfontfeatures{Scale=MatchLowercase}
  \defaultfontfeatures[\rmfamily]{Ligatures=TeX,Scale=1}
\fi
\usepackage{lmodern}
\ifPDFTeX\else  
    % xetex/luatex font selection
\fi
% Use upquote if available, for straight quotes in verbatim environments
\IfFileExists{upquote.sty}{\usepackage{upquote}}{}
\IfFileExists{microtype.sty}{% use microtype if available
  \usepackage[]{microtype}
  \UseMicrotypeSet[protrusion]{basicmath} % disable protrusion for tt fonts
}{}
\makeatletter
\@ifundefined{KOMAClassName}{% if non-KOMA class
  \IfFileExists{parskip.sty}{%
    \usepackage{parskip}
  }{% else
    \setlength{\parindent}{0pt}
    \setlength{\parskip}{6pt plus 2pt minus 1pt}}
}{% if KOMA class
  \KOMAoptions{parskip=half}}
\makeatother
\usepackage{xcolor}
\setlength{\emergencystretch}{3em} % prevent overfull lines
\setcounter{secnumdepth}{-\maxdimen} % remove section numbering
% Make \paragraph and \subparagraph free-standing
\makeatletter
\ifx\paragraph\undefined\else
  \let\oldparagraph\paragraph
  \renewcommand{\paragraph}{
    \@ifstar
      \xxxParagraphStar
      \xxxParagraphNoStar
  }
  \newcommand{\xxxParagraphStar}[1]{\oldparagraph*{#1}\mbox{}}
  \newcommand{\xxxParagraphNoStar}[1]{\oldparagraph{#1}\mbox{}}
\fi
\ifx\subparagraph\undefined\else
  \let\oldsubparagraph\subparagraph
  \renewcommand{\subparagraph}{
    \@ifstar
      \xxxSubParagraphStar
      \xxxSubParagraphNoStar
  }
  \newcommand{\xxxSubParagraphStar}[1]{\oldsubparagraph*{#1}\mbox{}}
  \newcommand{\xxxSubParagraphNoStar}[1]{\oldsubparagraph{#1}\mbox{}}
\fi
\makeatother


\providecommand{\tightlist}{%
  \setlength{\itemsep}{0pt}\setlength{\parskip}{0pt}}\usepackage{longtable,booktabs,array}
\usepackage{calc} % for calculating minipage widths
% Correct order of tables after \paragraph or \subparagraph
\usepackage{etoolbox}
\makeatletter
\patchcmd\longtable{\par}{\if@noskipsec\mbox{}\fi\par}{}{}
\makeatother
% Allow footnotes in longtable head/foot
\IfFileExists{footnotehyper.sty}{\usepackage{footnotehyper}}{\usepackage{footnote}}
\makesavenoteenv{longtable}
\usepackage{graphicx}
\makeatletter
\newsavebox\pandoc@box
\newcommand*\pandocbounded[1]{% scales image to fit in text height/width
  \sbox\pandoc@box{#1}%
  \Gscale@div\@tempa{\textheight}{\dimexpr\ht\pandoc@box+\dp\pandoc@box\relax}%
  \Gscale@div\@tempb{\linewidth}{\wd\pandoc@box}%
  \ifdim\@tempb\p@<\@tempa\p@\let\@tempa\@tempb\fi% select the smaller of both
  \ifdim\@tempa\p@<\p@\scalebox{\@tempa}{\usebox\pandoc@box}%
  \else\usebox{\pandoc@box}%
  \fi%
}
% Set default figure placement to htbp
\def\fps@figure{htbp}
\makeatother

\KOMAoption{captions}{tableheading}
\makeatletter
\@ifpackageloaded{caption}{}{\usepackage{caption}}
\AtBeginDocument{%
\ifdefined\contentsname
  \renewcommand*\contentsname{Table of contents}
\else
  \newcommand\contentsname{Table of contents}
\fi
\ifdefined\listfigurename
  \renewcommand*\listfigurename{List of Figures}
\else
  \newcommand\listfigurename{List of Figures}
\fi
\ifdefined\listtablename
  \renewcommand*\listtablename{List of Tables}
\else
  \newcommand\listtablename{List of Tables}
\fi
\ifdefined\figurename
  \renewcommand*\figurename{Figure}
\else
  \newcommand\figurename{Figure}
\fi
\ifdefined\tablename
  \renewcommand*\tablename{Table}
\else
  \newcommand\tablename{Table}
\fi
}
\@ifpackageloaded{float}{}{\usepackage{float}}
\floatstyle{ruled}
\@ifundefined{c@chapter}{\newfloat{codelisting}{h}{lop}}{\newfloat{codelisting}{h}{lop}[chapter]}
\floatname{codelisting}{Listing}
\newcommand*\listoflistings{\listof{codelisting}{List of Listings}}
\makeatother
\makeatletter
\makeatother
\makeatletter
\@ifpackageloaded{caption}{}{\usepackage{caption}}
\@ifpackageloaded{subcaption}{}{\usepackage{subcaption}}
\makeatother

\usepackage{bookmark}

\IfFileExists{xurl.sty}{\usepackage{xurl}}{} % add URL line breaks if available
\urlstyle{same} % disable monospaced font for URLs
\hypersetup{
  pdftitle={On direct systems of implications with graded attributes},
  pdfauthor={Manuel Ojeda-Hernández; Domingo López-Rodríguez},
  pdfkeywords={Fuzzy attributes, Implicational systems, Directness},
  colorlinks=true,
  linkcolor={blue},
  filecolor={Maroon},
  citecolor={Blue},
  urlcolor={Blue},
  pdfcreator={LaTeX via pandoc}}


\title{On direct systems of implications with graded attributes}
\author{Manuel Ojeda-Hernández \and Domingo López-Rodríguez}
\date{2025-01-28}

\begin{document}
\maketitle
\begin{abstract}
In this paper the problem of defining direct systems of implications in
the fuzzy setting is studied. The directness of systems allows a quick
computation of the closure operator in cases such as Fuzzy Formal
Concept Analysis. Characterizing these properties in algebraic terms is
deeply linked to Simplification Logic. After the theoretical results,
some thoughts on algorithms to provide direct systems are also
considered.
\end{abstract}


\newcommand{\el}[2]{#1_{#2}}

\section{Introduction}\label{introduction}

This document serves as supplementary material for the paper \emph{On
direct systems of implications with graded attributes} submitted to
EUSFLAT 2025.

\section{Detailed example}\label{detailed-example}

Let us consider a fuzzy formal context \(\mathbb K = (G, M, I)\), where
the set of attributes is \(M=\{a,b,c,d\}\), and such that the valuation
lattice, i.e., the lattice \(L\) such that \(I\in L^{G\times M}\), is
\(L=\{0, 0.5, 1\}\), equipped with the Lukasiewicz logical structure.

Take the system of implications:

\[\Sigma = \{a_{0.5}\,c \to b\,d_{0.5}, \quad a_{0.5}\,b\,d_{0.5} \to a\,c\,d\}\]

We will use the \texttt{DirectSystem} algorithm to construct a direct
system \(\Sigma_d\) equivalent to \(\Sigma\).

\subsection{Iterations}\label{iterations}

\subsubsection{Iteration 1}\label{iteration-1}

We will describe this iteration in detail, the following ones will be
more concise.

\textbf{Derived implications}

The algorithm loops over all pairs of implications, computing the
\emph{derived implication} when needed (required by the fuzzy exchange
condition). In this first iteration, only the two implications are
checked, for all \(\alpha,\beta\in L\) such that the requirements are
met:

Taking \begin{align*}
A\to B \quad& = \quad a_{0.5}\,b\,d_{0.5} \to a\,c\,d \\
C\to D \quad& = \quad a_{0.5}\,c \to b\,d_{0.5},
\end{align*} the execution of \texttt{AddDerived} produces:

\begin{itemize}
\tightlist
\item
  for \(\alpha=0.5, \beta=1\), the implication
  \(b_{0.5}\,c \to b\,d_{0.5}\).
\end{itemize}

Reversing the order in which the implications are considered, i.e.,
\begin{align*}
A\to B \quad& =  \quad a_{0.5}\,c \to b\,d_{0.5} \\
C\to D \quad& = \quad a_{0.5}\,b\,d_{0.5} \to a\,c\,d,
\end{align*} \texttt{AddDerived} provides:

\begin{itemize}
\tightlist
\item
  for \(\alpha=0.5, \beta=1\): the implication
  \(a_{0.5}\,b\,c_{0.5}\,d_{0.5} \to a\,c\,d\).
\item
  for \(\alpha=1, \beta=1\): the implication \(a_{0.5}\,c \to a\,d\).
\end{itemize}

This produces
\[\mathcal D = \{b_{0.5}\,c \to b\,d_{0.5},\quad a_{0.5}\,b\,c_{0.5}\,d_{0.5} \to a\,c\,d,\quad a_{0.5}\,c \to a\,d\}\]

\textbf{Combination phase}

The result of applying \texttt{Combine} to \(\Sigma\) and \(\mathcal D\)
follows these steps:

\begin{itemize}
\tightlist
\item
  Add the implication \(b_{0.5}\,c \to b\,d_{0.5}\) to \(\Sigma\), since
  there is no implication in \(\Sigma\) with the same left-hand side.
\item
  Analogously, add the implication
  \(a_{0.5}\,b\,c_{0.5}\,d_{0.5} \to a\,c\,d\) to \(\Sigma\).
\item
  Update \(a_{0.5}\,c \to b\,d_{0.5} \in \Sigma\) with
  \(a_{0.5}\,c \to a\,d\in\mathcal D\), to obtain the implication
  \(a_{0.5}\,c \to a\,b\,d\).
\end{itemize}

The value of the variable \texttt{change} returned by \texttt{Combine}
is \texttt{true} since there have been modifications to \(\Sigma\).

\textbf{Result of the iteration}

\(\Sigma\) after this iteration:
\[\Sigma = \{a_{0.5}\,c \to a\,b\,d,\quad a_{0.5}\,b\,d_{0.5} \to a\,c\,d,\quad b_{0.5}\,c \to b\,d_{0.5},\quad a_{0.5}\,b\,c_{0.5}\,d_{0.5} \to a\,c\,d\}\]

\subsubsection{Iteration 2}\label{iteration-2}

Now, we will summarise the steps, for the sake of readability.

\textbf{Derived implications}

\begin{align*}
\mathcal D = & \big\{b_{0.5}\,c \to a\,b\,d,\quad c_{0.5} \to a_{0.5}\,b_{0.5}\,d_{0.5},\quad a_{0.5}\,c \to a\,b\,d,\quad a_{0.5}\,b_{0.5}\,c \to a\,b\,d, \\
& \quad b\,c_{0.5} \to a\,c\,d,\quad a_{0.5}\,b\,c_{0.5}\,d_{0.5} \to a\,c\,d,\quad b_{0.5} \to a_{0.5}\,c_{0.5}\,d_{0.5}, \quad b \to a\,c\,d,\\
& \quad c \to b\,d_{0.5},\quad a_{0.5}\,b\,d_{0.5} \to a\,c\,d\big\}
\end{align*}

\textbf{Combination phase}

\begin{itemize}
\tightlist
\item
  Update \(b_{0.5}\,c \to b\,d_{0.5}\in\Sigma\) with
  \(b_{0.5}\,c \to a\,b\,d\in\mathcal D\) to obtain
  \(b_{0.5}\,c \to a\,b\,d\).
\item
  Add \(c_{0.5} \to a_{0.5}\,b_{0.5}\,d_{0.5}\),
  \(a_{0.5}\,b_{0.5}\,c \to a\,b\,d\), \(b\,c_{0.5} \to a\,c\,d\),
  \(b_{0.5} \to a_{0.5}\,c_{0.5}\,d_{0.5}\), \(b \to a\,c\,d\) and
  \(c \to b\,d_{0.5}\) to \(\Sigma\).
\end{itemize}

Therefore, the variable \texttt{change} is again \texttt{true} and a new
iteration is needed.

\textbf{Result of the iteration}

\(\Sigma\) after this iteration: \begin{align*}
\Sigma = & \big\{a_{0.5}\,c \to a\,b\,d,\quad a_{0.5}\,b\,d_{0.5} \to a\,c\,d,\quad b_{0.5}\,c \to a\,b\,d,\quad a_{0.5}\,b\,c_{0.5}\,d_{0.5} \to a\,c\,d,\\
& \quad c_{0.5} \to a_{0.5}\,b_{0.5}\,d_{0.5},\quad a_{0.5}\,b_{0.5}\,c \to a\,b\,d,\quad b\,c_{0.5} \to a\,c\,d,\quad b_{0.5} \to a_{0.5}\,c_{0.5}\,d_{0.5},\\
& \quad b \to a\,c\,d,\quad c \to b\,d_{0.5}\big\}.
\end{align*}

\subsubsection{Iteration 3}\label{iteration-3}

\textbf{Derived implications}

\begin{align*}
\mathcal D = & \big\{c_{0.5} \to a_{0.5}\,b_{0.5}\,d_{0.5},\quad  c \to a\,b\,d,\quad b_{0.5}\,c \to a\,b\,d,\quad a_{0.5}\,b_{0.5}\,c \to a\,b\,d, \\
& \quad a_{0.5}\,c \to a\,b\,d,\quad b\,c_{0.5} \to a\,c\,d,\quad b_{0.5} \to a_{0.5}\,c_{0.5}\,d_{0.5},\quad a_{0.5}\,b\,d_{0.5} \to a\,c\,d,\\
&\quad b \to a\,c\,d,\quad a_{0.5}\,b\,c_{0.5}\,d_{0.5} \to a\,c\,d\big\}
\end{align*}

\textbf{Combination phase}

\begin{itemize}
\tightlist
\item
  Update \(c \to b\,d_{0.5}\in\Sigma\) with \(c \to a\,b\,d\) to obtain
  \(c \to a\,b\,d\).
\end{itemize}

For this reason, \texttt{change} is set to \texttt{true}.

\textbf{Result of the iteration}

The implication system obtained so far is

\begin{align*}
\Sigma = & \big\{a_{0.5}\,c \to a\,b\,d,\quad a_{0.5}\,b\,d_{0.5} \to a\,c\,d,\quad b_{0.5}\,c \to a\,b\,d,\quad a_{0.5}\,b\,c_{0.5}\,d_{0.5} \to a\,c\,d,\\
&\quad c_{0.5} \to a_{0.5}\,b_{0.5}\,d_{0.5},\quad a_{0.5}\,b_{0.5}\,c \to a\,b\,d,\quad b\,c_{0.5} \to a\,c\,d,\quad b_{0.5} \to a_{0.5}\,c_{0.5}\,d_{0.5}, \\
&\quad b \to a\,c\,d,\quad c \to a\,b\,d\big\}
\end{align*}

\subsubsection{Iteration 4}\label{iteration-4}

\textbf{Derived implications}

\begin{align*}
\mathcal D = & \big\{c_{0.5} \to a_{0.5}\,b_{0.5}\,d_{0.5},\quad c \to a\,b\,d,\quad b\,c_{0.5} \to a\,c\,d,\\
&\quad a_{0.5}\,c \to a\,b\,d,\quad b_{0.5}\,c \to a\,b\,d,\quad a_{0.5}\,b_{0.5}\,c \to a\,b\,d\big\}
\end{align*}

\textbf{Combination phase}

It can be observed that \(\mathcal D \subseteq \Sigma\), thus at this
point, \texttt{change} is set to \texttt{false} and there is no need to
iterate further.

\subsection{Final result}\label{final-result}

This is the direct system returned by the algorithm:

\begin{align*}
\Sigma_d = & \big\{a_{0.5}\,c \to a\,b\,d,\quad a_{0.5}\,b\,d_{0.5} \to a\,c\,d,\quad b_{0.5}\,c \to a\,b\,d,\quad a_{0.5}\,b\,c_{0.5}\,d_{0.5} \to a\,c\,d,\\
&\quad c_{0.5} \to a_{0.5}\,b_{0.5}\,d_{0.5},\quad a_{0.5}\,b_{0.5}\,c \to a\,b\,d,\quad b\,c_{0.5} \to a\,c\,d,\quad b_{0.5} \to a_{0.5}\,c_{0.5}\,d_{0.5}, \\
&\quad b \to a\,c\,d,\quad c \to a\,b\,d\big\}
\end{align*}




\end{document}
